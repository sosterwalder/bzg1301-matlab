\begin{titlepage}


    \clearpage
    \vspace*{\fill}
    \begin{center}
        \begin{minipage}{.6\textwidth}
            \fontsize{26pt}{28pt}\selectfont
            Anhang
        \end{minipage}
    \end{center}
    \vfill % equivalent to \vspace{\fill}
    \clearpage


\end{titlepage}

\newpage 

% In den Anhang fügen Sie ein:
%  * Details des Projektpans, falls vorhanden
%  * Resultate und Zwischenresultate in Funktion der Projektiterationen
%  * Pflichtenheft / Anforderungsspezifikation (Stand Ende dritter Woche)
%  * Angaben zum Projektrepository
%  * Sitzungsprotokolle, falls vorhanden
%  * Weiterführende Erläuterungen zu den verwendeten Technologien, falls nötig
%  * Benutzerhandbuch, falls vorhanden und sinnvoll, es hier aufzulisten
%  * Installations- und Betriebsdokument, falls vorhanden und sinnvoll, es hier aufzulisten
% Unterlassen Sie das Anfügen von Listings.

\appendix

\section*{main.m}
\label{sec:appendix:main}
\lstinputlisting[language=Matlab,style=Matlab-editor]{appendix/main.m}

\newpage

\section*{Snake.m}
\label{sec:appendix:snake}
\lstinputlisting[language=Matlab,style=Matlab-editor]{appendix/Snake.m}

\newpage

\section*{SnakePoint.m}
\label{sec:appendix:snakepoint}
\lstinputlisting[language=Matlab,style=Matlab-editor]{appendix/SnakePoint.m}

\newpage

\section*{WindowUpdater.m}
\label{sec:appendix:windowupdater}
\lstinputlisting[language=Matlab,style=Matlab-editor]{appendix/WindowUpdater.m}

\newpage

\section*{test.m}
\label{sec:appendix:test}
\lstinputlisting[language=Matlab,style=Matlab-editor]{appendix/test.m}
