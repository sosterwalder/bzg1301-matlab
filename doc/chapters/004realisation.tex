\chapter{Material und Methoden}
\label{chap:realisation}

% Was genau wurde gemacht?

% In diesem Kapitel wird dargelegt, wie eine bestimmte Untersuchung durchgeführt wurde. Das Material (Pflanzen, Standorte, demographische Daten von Personen etc.), die Beobachtungs- oder Versuchspläne sowie genaue Informationen zu Messungen und zu statistischen Auswertungen werden beschrieben. Neue Methoden werden so beschrieben, dass die Leserinnen und Leser sie nachvollziehen können. Bekannte Methoden werden nur ganz kurz, mit Hinweisen auf die entsprechende Literatur, beschrieben.
% 
% Das Kapitel Material und Methoden gilt als unproblematisch, da der Inhalt gut bekannt ist. Es
% wird deshalb vorzugsweise als Erstes geschrieben.
% 
% \begin{itemize}
%     \item Die Leserinnen und Leser lernen die Datenbasis kennen und werden in die Lage versetzt, die Arbeit allenfalls zu wiederholen.
%     \item Bei einer Literaturarbeit, die die Ergebnisse anderer Arbeiten zusammenstellt und auswertet, sollte beschrieben werden, wie die verwendete Literatur recherchiert und ausgewertet wurde. Angaben über Suchstrategien und Datenbanken erleichtern das Nachvollziehen der Recherche.
%     \item Es muss deutlich werden, wie die Fragen der Einleitung beantwortet werden.
% \end{itemize}
Um die Erkennung von aktiven Konturen eines Bildes mittels Matlab umzusetzen, wurde eine Applikation in Matlab erstellt, welche es erlaubt ein Bild in bestimmte Regionen anhand aktiver Konturen zu unterteilen.

Wie~\citet[S. 147]{hudritsch:script:cp} schreibt, ist das Lösen der unter~\ref{chap:basics} beschriebenen Energiefunktion mittels Differnzialgleichung relativ umständlich. Es wird eine einfachere, diskrete Lösung mittels dem Algorithmus von Williams und Shah empfohlen.
