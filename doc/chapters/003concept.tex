\chapter{Konzept}
\label{chap:concept}

% Wie wurde vorgegangen? Gab es Alternativen?

Nachfolgend wird das Vorgehen anhand Zeitplanes und Meilensteinen beschrieben.\\
Jeder Meilenstein entspricht einem Eintrag im Zeitplan.

\section{Zeitplan}
\label{sec:timetable}

\begin{figure}[H]
    \begin{ganttchart}[
        vgrid,
        x unit=1cm,
        bar/.append style={fill=bfhgrey!50},
    ]{1}{10}
        \gantttitle{2014}{8}
        \gantttitle{2015}{2} \\
        \gantttitlelist{7,...,16}{1} \\ % chktex 11
        \ganttbar{Überlegung Thematik}{1}{1} \\
        \ganttbar{Präsentation Thematik}{2}{2} \\
        \ganttbar{Konzept}{3}{5}
        \ganttbar{}{7}{7}
        \ganttbar{}{9}{9} \\
        \ganttbar{Umsetzung GUI}{4}{5}
        \ganttbar{}{8}{8} \\
        \ganttbar{Implementation}{6}{8} \\
        \ganttbar{Präsentation Ergebnisse}{10}{10}

        \ganttlink{elem0}{elem1}
        \ganttlink{elem1}{elem2}
        \ganttlink{elem2}{elem3}
        \ganttlink{elem3}{elem4}
        \ganttlink{elem4}{elem8}
    \end{ganttchart}
    \caption{Zeitplan; Der Titel stellt Jahreszahlen, der Untertitel Semesterwochen dar}
\end{figure}

\section{Meilensteine}
\label{sec:milestones}

\subsection{Überlegung Thematik}
\label{subsec:topicsearch}
\paragraph{Semesterwoche 7}
Grundlegende Überlegungen zu möglichen Themen. Rascher Entscheid für ``Aktive Konturen''. Im Schwerpunkt seines Studiums (Computer Perception and Virtual Reality) hatte der Autor ``Aktive Konturen'' kennengelernt. Er wollte seine Kenntnisse mit dieser Arbeit vertiefen. MATLAB eignet sich besonders für Bildverarbeitung, wozu auch die aktive Konturen zählen.

\subsection{Präsentation Thematik}
\label{subsec:topicpres}
\paragraph{Semesterwoche 8}
Im Unterricht des Faches \textit{BZG1301: Programmierung in Matlab/Octave} stellten die Teilnehmer ihre Themen vor. Herr Prof.\ Marx Stampfli war mit der gewählten Thematik des Autors einverstanden.

\subsection{Konzept}
\label{subsec:concept}
\paragraph{Semesterwochen 9 bis 11, 13 und 15}
Konzeptgestaltung der hier vorliegenden Projektarbeit. Kontinuierliche Erweiterung in den folgenden Semsterwochen.

\subsection{Umsetzung GUI}
\label{subsec:gui}
\paragraph{Semesterwochen 10 bis 11, 14}
Erstellung Prototyp der grafischen Benutzeroberfläche. Initial ohne Funktion, schrittweise Weiterentwicklung.

\subsection{Implementation}
\label{subsec:impl}
\paragraph{Semesterwochen 12 bis 14}
Umsetzung des Algorithmus zur Erkennung von aktiven Konturen in MATLAB.\ Initial statische Umsetzung anhand fixer Parameter und eines fixen Bildes. Anschliessend schrittweise Verknüpfung der Implementation mit der grafischen Benutzeroberfläche.
