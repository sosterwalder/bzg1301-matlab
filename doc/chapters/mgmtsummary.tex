\chapter{Zusammenfassung}
\label{chap:summary}
% Die Zusammenfassung wird am zweithäufigsten gelesen. Danach entscheiden viele Leute, ob sie den Bericht weiterlesen oder nicht. In der Zusammenfassung werden Fragestellung und Hypothesen, Methoden, Ergebnisse und Schlussfolgerungen möglichst kurz dargestellt (jeweils etwa zwei bis drei Sätze).

Diese Projektarbeit untersucht, wie gut sich Regionen anhand aktiver Konturen finden lassen. Dies im Gegensatz zu den anderen, bereits in MATLAB verfügbaren Methoden.

Dazu wurde ein Prototyp einer Applikation zur Erkennung aktiver Konturen in Bildern mit Hilfe von MathWorks \gls{MATLAB}~\footnote{\url{http://www.mathworks.com/products/matlab/}} erstellt.

Damit liess sich in einem rauschbehafteten Röntgenbild die definierte Region als zusammenhängend erkennnen.

Ein Vergleich der Lösungen ist unter den gegebenen Umständen jedoch nicht realistisch: Während mittels der MATLAB-internen Methode mehrere Regionen eines Bildes extrahiert werden können, ist dies mit der selbst implementierten Variante zum jetzigen Zeitpunkt nicht möglich.
