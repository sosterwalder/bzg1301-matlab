\chapter{Einleitung}
\label{chap:intro}

Diese Projektarbeit ist Teil eines Projektes im Rahmen des Moduls \textit{BZG1301: Programmierung in Matlab/Octave} der Berner Fachhochschule für angewandte Wissenschaft.

Das Ziel dieser Projektarbeit ist die Umsetzung eines Prototypen einer Applikation mithilfe von MathWorks Matlab~\footnote{\url{http://www.mathworks.com/products/matlab/}} zur Erkennung \gls{aktiver Konturen} in Bildern.

Dabei handelt es sich um eine Möglichkeit ``um in verrauschten Bildern Konturen von Regionen zu bestimmen, die mit anderen \gls{Segmentationsmethoden} nur bedingt als zusammenhängend erkannt werden.''~\cite[S. 144]{hudritsch:script:cp}

Bei dieser Projektarbeit soll untersucht werden, wie gut sich Regionen anhand aktiver Konturen im Gegensatz zu den herkömmlichen, bereits in Matlab verfügbaren Methoden, finden lassen. Als Daten kommen dabei im Internet gesammelte Bilder zum Einsatz. Es wird erwartet, dass die Regionen, vorallem in verrauschten Bildern, mittels den aktiven Konturen durchgehend als zusammenhängend erkannt werden.

%  Die Einleitung führt die Leserinnen und Leser in die Thematik der Arbeit ein. Sie setzt den Rahmen für die Arbeit, skizziert den aktuellen Wissensstand und geht auf die konkreten Fragestellungen und Ziele der vorliegenden Arbeit ein. Aus der Einleitung muss klar hervorgehen, weshalb die Arbeit gemacht wurde und welche Bedeutung ihr zukommt.
%  
%  \begin{itemize}
%      \item Ziel, worum geht es?
%      \item Eine gute Einleitung motiviert, den ganzen Bericht zu lesen.
%      \item Eine gute Einleitung enthält alle später angesprochenen Themen, aber nur diese!
%      \item Was will ich wissen?
%      \item Was habe ich für Daten? (Wie gesammelt, woher, Weiterarbeit?)
%      \item Welche Modelle brauche ich? (Matlab Toolbox)
%      \item Was erwarte ich von den Resultaten?
%  \end{itemize}
