\chapter{Einleitung}
\label{chap:intro}

Diese Projektarbeit ist im Rahmen des Moduls \textit{BZG1301: Programmierung in Matlab/Octave} der Berner Fachhochschule für angewandte Wissenschaft entstanden.

Das Ziel dieser Projektarbeit ist die Erarbeitung eines Prototypen für eine Applikation zur Erkennung \glslink{aktiver Konturen}{aktiver Konturen} in Bildern mittels MathWorks \gls{MATLAB}~\footnote{\url{http://www.mathworks.com/products/matlab/}}.

Dabei handelt es sich um eine von mehreren Möglichkeiten ``\ldots in verrauschten Bildern Konturen von Regionen zu bestimmen, die mit anderen \glslink{Segmentationsmethoden}{Segmentationsmethoden} nur bedingt als zusammenhängend erkannt werden.''~\cite[S. 144]{hudritsch:script:cp}

In dieser Projektarbeit soll untersucht werden, wie gut sich Regionen anhand aktiver Konturen im Gegensatz zu den anderen, bereits in MATLAB verfügbaren Methoden, finden lassen. Als Daten kommen im Internet gesammelte Bilder zum Einsatz.\\
Ziel ist, Regionen mittels aktiver Konturen in verrauschten Bildern als zusammenhängend zu erkennen.

%  Die Einleitung führt die Leserinnen und Leser in die Thematik der Arbeit ein. Sie setzt den Rahmen für die Arbeit, skizziert den aktuellen Wissensstand und geht auf die konkreten Fragestellungen und Ziele der vorliegenden Arbeit ein. Aus der Einleitung muss klar hervorgehen, weshalb die Arbeit gemacht wurde und welche Bedeutung ihr zukommt.
%  
%  \begin{itemize}
%      \item Ziel, worum geht es?
%      \item Eine gute Einleitung motiviert, den ganzen Bericht zu lesen.
%      \item Eine gute Einleitung enthält alle später angesprochenen Themen, aber nur diese!
%      \item Was will ich wissen?
%      \item Was habe ich für Daten? (Wie gesammelt, woher, Weiterarbeit?)
%      \item Welche Modelle brauche ich? (MATLAB Toolbox)
%      \item Was erwarte ich von den Resultaten?
%  \end{itemize}
