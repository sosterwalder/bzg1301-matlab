\chapter{Grundlagen}
\label{chap:basics}

% Was muss man als technischen Hintergrund wissen?

Bei aktiven Konturen handelt es sich um eine Art der Segmentierung, d.h.\ ein Verfahren der Bildanalyse. Dabei wird ein Bild in interessante und uninteressante Regionen unterteilt. Man weist jedes Pixel eines Bildes einer bestimmten Region zu.

Die aktiven Konturen sind ein Verfahren der modellbasierten Segmentierung. Die Grenzen der Segmente müssen im Voraus bestimmt werden.

(vgl. \citeauthor*[S. 133 und 144]{hudritsch:script:cp}) % chktex 10 chktex 17 chktex 9

Eine aktive Kontur wird mit Hilfe von \glslink{parametrisierten}{parametrisierten}, geschlossenen Kurven beschrieben.

\begin{figure}[H]
    \begin{align}
        r(s) & = (x(s), y(s)), s \in [0,1]
    \end{align}
    \caption{Position einer aktiven Kontur als parametrische Darstellung,~\cite{kass88snakes:active}}
\end{figure}

Zur Beschreibung aktiver Konturen kommen häufig B-Spline-Kurven zum Einsatz. Sie haben den Vorteil aus mehreren (Kurven-) Segmenten zu bestehen, welche ihrerseits mit einigen wenigen Kontrollpunkten definiert sind. Die Kontrollpunkte sind zugleich die Koeffizienten der B-Spline-Basisfunktion. Der Grad der Kurve ist unabhängig von der Anzahl der Kontrollpunkte. (vgl.~\citeauthor*[S. 79]{fuhrer:script:splines}) % chktex 10 chktex 17 chktex 9

Die Basis einer aktiven Kontur bildet eine \glslink{B-Spline-Kurve}{B-Spline-Kurve}. Ihre \glslink{Kontinuität}{Kontinuität} wird von den Kräften eines Bildes und externen Kräften bestimmt. Die internen Kräfte der Kurve ermöglichen eine schrittweise Glättung der Kontur. Die Kräfte eines Bildes ziehen die Kontur zu gewissen Merkmalen des Bildes hin, wie z.B. Linien oder Kanten. Durch die externen Kräfte wird die Kontur einem gewünschten lokalen Minimum angenähert.\\
(vgl.~\citeauthor*[S. 323]{kass88snakes:active}) % chktex 10 chktex 17 chktex 9

Bei parametrischer Darstellung der Position einer aktiven Kontur lässt sich aus den genannten Kräften folgende Energiefunktion ableiten:

\begin{figure}[H]
    \begin{align}
        E_{snake}^* & = \int_0^1 E_{snake}(v(s))\, \mathrm{d}s\\
         & = \int_0^1 E_{int}(v(s)) + E_{image}(v(s)) + E_{con}(v(s))\, \mathrm{d}s
    \end{align}
    \caption{Energiefunktion einer aktiven Kontur,~\cite{kass88snakes:active}}
\end{figure}

Auf die Detailbeschreibung der einzelnen Teile der Energiefunktion wurde verzichtet, um den Umfang dieser Arbeit einzuschränken. Details bei~\citet[S. 247]{hudritsch:script:cp} und bei~\citet[S. 323 bis 328]{kass88snakes:active}.  % chktex 10 chktex 17 chktex 9
